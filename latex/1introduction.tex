 % -*- root: thesis.tex -*-
\chapter{Introduction}
\label{chapter:intro}

\section{Background}
\label{section:background}

Futurice is a software consult agency developing digital solutions for their clients. Their methodology has alwasy been agile and has long focused on the overall experience of the end-users. They aim to take their customers through the whole lifecycle of a product, from ideation to lifecycle management. The process Futurice uses for this holistic approach is called Lean Service Creation (LSC). An integral part of LSC is workshops held with customers.

When workshops are held they are about some topic. This can be purely fictional or based on actual business goals or needs. Using fictional topics may lead to more creative thinking as participants are not locked in by previous knowledge about what can and can not be done, which may be the case when the workshop is held about somethign closely related to the participants' own area of expertice. One way to prevent this is to have sessions where participants are not allowed to take known limitations into account.

Another problem for workshops could also be that participants do not know about all the possibilities offered by new technologies. Especially technologies that are not directly related to their competences may be overlooked. This can be harmful for new innovations as innovations happen when different technologies are meet.





\section{Research problem and question}
\label{section:problem}

How do you conduct productive service creation workshops?

How could you utilize device to inspire ideation during workshops?

\section{Scope}
\label{section:scope}

The scope of this thesis covers some best practices of workshops held in a service creation context. The second part of the thesis covers the findings of workhops held with devices and catalysts for innovation.


\section{Structure of the Thesis}
\label{section:structure}
