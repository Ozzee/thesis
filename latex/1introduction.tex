 % -*- root: thesis.tex -*-
\chapter{Introduction}
\label{chapter:intro}

\section{Background}
\label{section:background}

Futurice is a software consult agency developing digital solutions for their clients. Their methodology has alwasy been agile and has long focused on the overall experience of the end-users. They aim to take their customers through the whole lifecycle of a product, from ideation to lifecycle management. The process Futurice uses for this holistic approach is called Lean Service Creation (LSC).


\section{Research problem and question}
\label{section:problem}

The research problem is defined as follows:\\

\textit{How does a new service creation project that uses new and untested technology differ from a traditional service creation project that uses familiar technologies and concepts?}\\

To investigare this problem three research questions have been set up in table \ref{tbl:questions}.


\begin{table}
  \begin{tabular}{p{200pt} | p{70pt} | p{70pt}}
    Question & Litterature review & Empirical study \\
    \hline
    What are the currently available frameworks for new service creation? & x & \\
    What are the success factors of a new service creation project involving novel technology? &  & x \\
    What are the challenges when executing a new service creation project using novel technology? &  & x \\
  \end{tabular}
  \caption{Research questions and their respective sections}
  \label{tbl:questions}
\end{table}

\section{Scope}
\label{section:scope}




\section{Structure of the Thesis}
\label{section:structure}

The thesis is split into two distinct sections. First, chapter \ref{chapter:litterature} covers service design and new service creation frameworks. Second, chapter \ref{chapter:empirical} covers the findings of a project involving new service creation using new and untested technology for end-users.


