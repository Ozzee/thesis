 % -*- root: thesis.tex -*-

\chapter{Litterature review}
\label{chapter:litterature}

In this section I will present the existing litterature and what general guidelines or conclusions it presents. I will refer to the main concepts presented in books and compare those findings to case studies that have implemented the relevant concepts in order to find some commonalities if there are any.

I will also present the process for finding and analyzing the material I have chosen.

\begin{itemize}
  \item{How was the litterature review conducted?}
  \begin{itemize}
    \item{Scholar}
    \item{Scholar through related articles of known good articles}
    \item{Key articles then their sources}
    \item{Same author, look for more}
    \item{Same conference/journal, look for more}
  \end{itemize}
  \
\end{itemize}

Keywords: \textit{new service creation, digital service creation, service-dominant, design thinking, new service development}

\section{The origin of lean thinking}
\label{section:leanorigins}


Even though the traditional wisdom is that the Japanese car manufacturers had a significant advantage over western competitors due to the lean methodologies they implemented there is some controcersy over wether this was in fact the case. Dyb� \& Sharp argue that by examining the facts and taking automation into account the Japanese did not have a superior organizational advantage. \cite{Dyba2012WhatS}

\section{Lean versus Agile}
\label{section:leanvsagile}

\fixme{
Lean should be thought of a set of principles rather than practices. This article has some excellent points and trends to talk about.\cite{Poppendieck2012Lean}
}

End with something about everything being services nowdays and lean/agile being developer focused. The continue to...

\section{Lean Service Creation}
\label{section:lsc}

\fixme{Lean Service Creation is about build measure learn and create somehing new. Tie this into the problem and focus on the new part. Using news devices maybe?}

Lean Service Creation is based on the ideas of ``The Lean Startup'' as described by Eric Reis in his 2011 book.\cite{ries2011lean}

\fixme{How do I get a source for this? Interview?}

\section{Comparison of existing litterature}

In this section I will compare articles regarding case studies focusing on lean software development/management.

Two case studies. This is an early attempt to test lean software development. The article covers an experiment set up to test the validity of using lean principles in software development. \cite{Middleton2001Lean}

Timberline Inc. case study. Probably the first case of adopting lean principles to software development.\cite{Middleton2005Lean}

BBC Worldwide case study. The gist of it is that the performance of the team improved when adopting lean practices, but there were some challenges in fitting the the lean principles with the rest of the company.\cite{Middleton2012Lean}


\section{Research problem and question}
\label{section:problem}

This chapter will end with the research problem and questions.

The research problem is defined as follows:\\

\textit{What are the commonly accepted and used lean software development practices and how do they change when working with new and unfamiliar technology?}\\

To investigare this problem three research questions have been set up in table \ref{tbl:questions}.


\begin{table}
  \begin{tabular}{p{200pt} | p{70pt} | p{70pt}}
    Question & Litterature review & Empirical study \\
    \hline
    What are the currently available best practices for lean software projects? & x & \\
    Which of these best practices need to be adapted when working with new technology? &  & x \\
    How do these best practices need to adapted? &  & x \\
    Which best practices remain valid? &  & x \\
  \end{tabular}
  \caption{Research questions and their respective sections}
  \label{tbl:questions}
\end{table}