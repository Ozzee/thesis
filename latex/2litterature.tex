 % -*- root: thesis.tex -*-

\chapter{Litterature review}
\label{chapter:litterature}

In this section I will present the existing litterature and what general guidelines or conclusions it presents. I will refer to the main concepts presented in books and compare those findings to case studies that have implemented the relevant concepts in order to find some commonalities if there are any.

I will also present the process for finding and analyzing the material I have chosen.

\begin{itemize}
  \item{How was the litterature review conducted?}
  \begin{itemize}
    \item{Scholar}
    \item{Key articles then their sources}
    \item{Same author, look for more}
    \item{Same conference/journal, look for more}
  \end{itemize}
  \
\end{itemize}

Keywords: \textit{new service creation, digital service creation, service-dominant, design thinking, new service development}


\section{Research problem and question}
\label{section:problem}

This chapter will end with the research problem and questions.

The research problem is defined as follows:\\

\textit{What are the commonly accepted and used lean software development practices and how do they change when working with new and unfamiliar technology?}\\

To investigare this problem three research questions have been set up in table \ref{tbl:questions}.


\begin{table}
  \begin{tabular}{p{200pt} | p{70pt} | p{70pt}}
    Question & Litterature review & Empirical study \\
    \hline
    What are the currently available best practices for lean software projects? & x & \\
    Which of these best practices need to be adapted when working with new technology? &  & x \\
    How do these best practices need to adapted? &  & x \\
    Which best practices remain valid? &  & x \\
  \end{tabular}
  \caption{Research questions and their respective sections}
  \label{tbl:questions}
\end{table}