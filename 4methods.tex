 % -*- root: thesis.tex -*-

\chapter{Methods}
\label{chapter:methods}

% Introduction

This chapter presents the methods used to gather and analyze the data of this study. First, the research problem is reiterated and the motivation for the used methods are presented. Second, grounded theory, the chosen method of analysis, is described in short. Third, the way data was collected through interviews is presented. Fourth, the participants and their backgrounds are disclosed. Fifth, the process of analysis is explained using grounded theory. Finally, the methods are critically analyzed and their validity and credibility are evaluated.

\section{Methodological solution}
\label{section:methodsolution}

% Research problem and why this method suits it

The research problem defined in section \ref{section:problem} aims to understand the lean software development process. Choosing a method that focuses of understanding a phenomenon is preferable, as there is limited data about the research area. The most straightforward way of understanding the underlying culture of lean in this context is by interviews. A proven method is needed to analyze the data from the interviews. Grouping by themes found in literature, such as the principles presented in chapter \ref{chapter:literature}, is one option. A better option is to analyze the data as such, without preconceived notions of the results. The grounded theory is well suited for this purpose.

\section{Method of analysis}
\label{section:methodanalysis}

% Present grounded theory

Grounded theory was first presented by Glaser and Strauss in 1967. Their field was sociology and the aim of grounded theory was to enable qualitative researchers to formulate theories based on scientific methodology which would be relevant in that field. The method was aimed at sociologists, but the authors recognized that the theory could be useful for any domain looking to understand social phenomena using qualitative data. \cite{glaser1967discovery}

As the name suggests, grounded theory is grounded in the data. This means that when using grounded theory one starts by gathering data and analyzing it. This is different from the traditional approach in science where you first form a hypothesis and then verify or disprove the hypothesis by empirical study. This roundabout way of forming a theory has led some to calling the hypothesis formed by grounded theory as ``reverse engineered''. \cite{lazar2010research}

Various forms of empirical data can serve as the starting point for grounded theory research. In this research the source material is interviews, but GT can also be used with ethnography, observation, and case studies among others. \cite{lazar2010research}

The GT method is performed in four steps. The first step is coding. This is where the researchers identify phenomenon in the source material and gives them names or codes. In GT coding is done as open coding. This refers to the fact that the coding is done with an open mind, letting the concepts emerge from the source material and not from outside influences. The second step is developing concepts. This is where codes that describe similar phenomenon are grouped together to form concepts. In the third step, categories are grouped together to form categories. Finally, a theory is formed based on the causal connections between concepts and categories. \cite{lazar2010research}


\section{Data collection}
\label{section:methoddata}

% haastikset, semi structured

The empirical study was conducted as a series of interviews. Five interviews were conducted, each lasting between 50 and 65 minutes.

The interviews were conducted as semi-structured interviews. Two sets of questions were prepared, one set for Futurice employees and one for customer employees. The Futurice employee questions were modified slightly to better suit the expertise of the participant. The questions may be found in Appendix \ref{appendix:questions}. During the interviews participants were encouraged to answer with their own interpretations. Follow up questions were also asked based on the direction of the answers the participants gave.

The questions were prepared based on the lean principles presented in chapter \ref{chapter:literature}. Subjects were asked about previous experiences with software projects and with lean methods. The questions explored project based work as well as organizational culture. Questions were also left intentionally up for interpretation in order to map the subject's own interpretations and avoid unnecessary bias by the interviewer.

\fixme{Add questions appendix}

\section{Participants}
\label{section:methodparticipants}

The participants were from Futurice (3 participants) and Futurice's customer organizations (2 participants).

Participants were asked to sign a letter of informed consent before the interview. They were also told that they may end the interview at any time and choose not to respond to any and all questions. Furthermore, the participants had the right ask for any disclosed information to be removed from the final interview notes.

The interviews were recorded in audio format and detailed written notes were done based on the recordings. The interviewer agreed to archive the recordings responsibly and not share the recordings with any third party. The letter of informed consent can be studied in appendix \ref{appendix:letter}.

\fixme{Add letter of informed consent as appendix}

\section{Process of analysis}
\label{section:methodprocess}

% miten teemat syntyivät

Coding and categorization of the interviews was done using the Atlas.ti software. The written notes from the interviews were imported and codified. Then networks of concepts were modeled using the network functionality of the software.

Categories were chosen based on the data form the interviews. The data presented some clear themes which were identified. Some in-vivo categories were also present in the data, but most were general themes. It is safe to assume that the themes were influenced by the existing literature on lean, both based on the set of questions and the bias of the researcher.

\fixme{Write this again when the actual analysis has been done}

\section{Method evaluation}
\label{section:methodevaluation}

% credibility etc.
The preconceptions and background of the author were recognized and taken into account during the analysis process and special care was taken to analyze the data sensitively. However, it can still be assumed that the analysis is not entirely objective, as the researcher always affects the process.

Subjects were asked for clarifying questions and viewpoints when they seemed to veer too far off course. Clarifications about statements were asked to confirm the assumed meaning of answers which could have been interpreted by the subjects as ``the right answer'' and thus confirmed. The subjects were, however, informed repeatedly that there were no wrong answers and as such they could feel safe answering any way they wished. 


% The credibility of this study is at a satisfactory level, although it is safe to assume that the researcher influenced the interview situation. 
% <-- Need to rewrite credibility part -->

\fixme{TODO: Ask subjects to read and ``confirm'' analysis}

The low number of subjects and organizations studied makes the findings of the study specific and not generalizable. However, the results of the study serve as a good baseline for further research. The analysis of this study is valid for a specific set and can be used as a base for further study.

The interviews were conducted in one round. Ideally GT is done iteratively so that once one round of interviews has been analyzed the theory can be confirmed or refined by a second round of interviews. This study lacked the resources to conduct several iterations of interviews, which has to be taken into account when assessing the validity of the data.

Themes from the existing literature were clearly present in the answers of the subject. Some of the questions were specifically formulated to seek out answers related to the existing literature, but even those that were not supported the principals present in the existing literature. Existing literature supports the findings of the study.

