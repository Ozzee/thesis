\documentclass[12pt,a4paper,oneside,pdftex]{report}
\usepackage[utf8]{inputenc}
\usepackage{lmodern}
\usepackage[OT1]{fontenc}
\usepackage[finnish,swedish,english]{babel}


\usepackage[square,sort&compress,numbers]{natbib}
\usepackage{eurosym}

\usepackage{verbatim}

\usepackage{longtable}

\usepackage{subfigure}

\usepackage[medium]{titlesec}

\usepackage{tikz}
% You also need to specify which TikZ libraries you use
\usetikzlibrary{positioning}
\usetikzlibrary{calc}
\usetikzlibrary{arrows}
\usetikzlibrary{decorations.pathmorphing,decorations.markings}
\usetikzlibrary{shapes}
\usetikzlibrary{patterns}


% The aalto-thesis package provides typesetting instructions for the
% standard master's thesis parts (abstracts, front page, and so on)
% Load this package second-to-last, just before the hyperref package.
% Options that you can use:
%   mydraft - renders the thesis in draft mode.
%             Do not use for the final version.
%   doublenumbering - [optional] number the first pages of the thesis
%                     with roman numerals (i, ii, iii, ...); and start
%                     arabic numbering (1, 2, 3, ...) only on the
%                     first page of the first chapter
%   twoinstructors  - changes the title of instructors to plural form
%   twosupervisors  - changes the title of supervisors to plural form
\usepackage[mydraft, doublenumbering]{aalto-thesis}
%\usepackage[mydraft,doublenumbering]{aalto-thesis}
%\usepackage{aalto-thesis}


\RequirePackage[pdftex]{hyperref}
\hypersetup{colorlinks=false,raiselinks=false,breaklinks=true}
\hypersetup{pdfborder={0 0 0}}
\hypersetup{bookmarksnumbered=true}
\hypersetup{bookmarksopen=true,bookmarksopenlevel=1}

 % -*- root: thesis.tex -*-

\newcommand{\TITLE}{Lean Software Project Execution Through Consulting}
\newcommand{\FTITLE}{}
\newcommand{\STITLE}{}
\newcommand{\SUBTITLE}{}
\newcommand{\FSUBTITLE}{}
\newcommand{\SSUBTITLE}{}
\newcommand{\DATE}{December 1, 2015}
\newcommand{\FDATE}{1. joulukuuta 2015}
\newcommand{\SDATE}{Den 1 December 2015}

% Supervisors and instructors
% ------------------------------------------------------------------
% If you have two supervisors, write both names here, separate them with a
% double-backslash (see below for an example)
% Also remember to add the package option ``twosupervisors'' or
% ``twoinstructors'' to the aalto-thesis package so that the titles are in
% plural.
% Example of one supervisor:
%\newcommand{\SUPERVISOR}{Professor Antti Ylä-Jääski}
%\newcommand{\FSUPERVISOR}{Professori Antti Ylä-Jääski}
%\newcommand{\SSUPERVISOR}{Professor Antti Ylä-Jääski}
% Example of twosupervisors:
\newcommand{\SUPERVISOR}{Professor Marjo Kauppinen}
\newcommand{\FSUPERVISOR}{Professori Marjo Kauppinen}
\newcommand{\SSUPERVISOR}{Professor Marjo Kauppinen}

% If you have only one instructor, just write one name here
\newcommand{\INSTRUCTOR}{Suvi Uski (?)}
\newcommand{\FINSTRUCTOR}{Suvi Uski (?)}
\newcommand{\SINSTRUCTOR}{Suvi Uski (?)}
% If you have two instructors, separate them with \\ to create linefeeds
% \newcommand{\INSTRUCTOR}{Olli Ohjaaja M.Sc. (Tech.)\\
%  Elli Opas M.Sc. (Tech)}
%\newcommand{\FINSTRUCTOR}{Diplomi-insinööri Olli Ohjaaja\\
%  Diplomi-insinööri Elli Opas}
%\newcommand{\SINSTRUCTOR}{Diplomingenjör Olli Ohjaaja\\
%  Diplomingenjör Elli Opas}

% If you have two supervisors, it is common to write the schools
% of the supervisors in the cover page. If the following command is defined,
% then the supervisor names shown here are printed in the cover page. Otherwise,
% the supervisor names defined above are used.
\newcommand{\COVERSUPERVISOR}{Professor Marjo Kauppinen, Aalto University}

% The same option is for the instructors, if you have multiple instructors.
% \newcommand{\COVERINSTRUCTOR}{Olli Ohjaaja M.Sc. (Tech.), Aalto University\\
%  Elli Opas M.Sc. (Tech), Aalto SCI}


% Other stuff
% ------------------------------------------------------------------
\newcommand{\PROFESSORSHIP}{Software Engineering and Business}
\newcommand{\FPROFESSORSHIP}{Ohjelmistotuotanto ja -liiketoiminta}
\newcommand{\SPROFESSORSHIP}{Programvaruproduktion och -affärsverksamhet}
% Professorship code is the same in all languages
\newcommand{\PROFCODE}{T-76}
\newcommand{\KEYWORDS}{lean, lean software, lean software project, service creation, agile, LSC}
\newcommand{\FKEYWORDS}{}
\newcommand{\SKEYWORDS}{}
\newcommand{\LANGUAGE}{English}
\newcommand{\FLANGUAGE}{Englanti}
\newcommand{\SLANGUAGE}{Engelska}

% Author is the same for all languages
\newcommand{\AUTHOR}{Oskar Ehnström}

% Abstracts

\newcommand{\ABSTRACTEN}{

The current literature on lean software development is derived from the literature on lean manufacturing. These principles can be applied as various practices in software projects depending on the context. The principles hold well for the general domain of software development.

Lean enables dealing with change, which has been found to be crucial when dealing with an unfamiliar domain. Software development often deals with uncertain requirements as well as new and unfamiliar technology, especially when software development is not the core business of a company.

Introducing lean principles into an organization can be challenging. Utilizing outside consultants to handle projects dealing with uncertain requirements and technologies could enable a non-lean organization to benefit from lean principles without radical change to the organization.

This thesis studies how a software project implemented by a lean vendor can enable an organization to use lean methods by proxy. The organization could then reap the rewards of lean principles in the project without a significant change to their existing organization.

The study uses one software project as a case study. As such, general conclusions about the subject can not be drawn. However, the results indicate that using an outside vendor may indeed enable an organization to utilize lean principles in fast paced projects without significant change to the organization itself.

}

\newcommand{\ABSTRACTFI}{}
\newcommand{\ABSTRACTSV}{}



% Currently the English versions are used for the PDF file metadata
% Set the PDF title
\hypersetup{pdftitle={\TITLE\ \SUBTITLE}}
% Set the PDF author
\hypersetup{pdfauthor={\AUTHOR}}
% Set the PDF keywords
\hypersetup{pdfkeywords={\KEYWORDS}}
% Set the PDF subject
\hypersetup{pdfsubject={Master's Thesis}}


% Layout settings
% ------------------------------------------------------------------

% Bibliography style
% acm style gives you a basic reference style. It works only with numbered
% references.
\bibliographystyle{acm}
% Plainnat is a plain style that works with both numbered and name citations.
% \bibliographystyle{plainnat}


% Extra hyphenation settings
\hyphenation{di-gi-taa-li-sta yksi-suun-tai-sta}



% The preamble ends here, and the document begins.
% Place all formatting commands and such before this line.
% ------------------------------------------------------------------
\begin{document}
% This command adds a PDF bookmark to the cover page. You may leave
% it out if you don't like it...
\pdfbookmark[0]{Cover page}{bookmark.0.cover}
% This command is defined in aalto-thesis.sty. It controls the page
% numbering based on whether the doublenumbering option is specified
\startcoverpage

% Cover page
% ------------------------------------------------------------------
% Options: finnish, english, and swedish
% These control in which language the cover-page information is shown
\coverpage{english}


% Abstracts
% ------------------------------------------------------------------
% Include an abstract in the language that the thesis is written in,
% and if your native language is Finnish or Swedish, one in that language.

% Abstract in English
% ------------------------------------------------------------------
\thesisabstract{english}{\ABSTRACTEN}

% Abstract in Finnish
% ------------------------------------------------------------------
\thesisabstract{finnish}{\ABSTRACTFI}

% Abstract in Swedish
% ------------------------------------------------------------------
\thesisabstract{swedish}{\ABSTRACTSV}


% Acknowledgements
% ------------------------------------------------------------------
% Select the language you use in your acknowledgements
\selectlanguage{english}

% Uncomment this line if you wish acknoledgements to appear in the
% table of contents
%\addcontentsline{toc}{chapter}{Acknowledgements}

% The star means that the chapter isn't numbered and does not
% show up in the TOC
\chapter*{Acknowledgements}

TODO: Thank people here

\vskip 10mm

\noindent Espoo, \DATE
\vskip 5mm
\noindent\AUTHOR

% Acronyms
% ------------------------------------------------------------------
% Use \cleardoublepage so that IF two-sided printing is used
% (which is not often for masters theses), then the pages will still
% start correctly on the right-hand side.
%\cleardoublepage
% Example acronyms are placed in a separate file, acronyms.tex
%\input{acronyms}

% Table of contents
% ------------------------------------------------------------------
\cleardoublepage
% This command adds a PDF bookmark that links to the contents.
% You can use \addcontentsline{} as well, but that also adds contents
% entry to the table of contents, which is kind of redundant.
% The text ``Contents'' is shown in the PDF bookmark.
\pdfbookmark[0]{Contents}{bookmark.0.contents}
\tableofcontents

% List of tables
% ------------------------------------------------------------------
% You only need a list of tables for your thesis if you have very
% many tables. If you do, uncomment the following two lines.
% \cleardoublepage
% \listoftables

% Table of figures
% ------------------------------------------------------------------
% You only need a list of figures for your thesis if you have very
% many figures. If you do, uncomment the following two lines.
% \cleardoublepage
% \listoffigures

% The following label is used for counting the prelude pages
\label{pages-prelude}
\cleardoublepage

%%%%%%%%%%%%%%%%% The main content starts here %%%%%%%%%%%%%%%%%%%%%
% ------------------------------------------------------------------
% This command is defined in aalto-thesis.sty. It controls the page
% numbering based on whether the doublenumbering option is specified
\startfirstchapter

% Add headings to pages (the chapter title is shown)
\pagestyle{headings}

% The contents of the thesis are separated to their own files.
% Edit the content in these files, rename them as necessary.
% ------------------------------------------------------------------
 % -*- root: thesis.tex -*-

 % Tiimalasti rakenne

\chapter{Introduction}
\label{chapter:intro}


\begin{itemize}
  \item{Lean dictionary definition}
  \item{Where does lean come from?}
  \item{Why was lean manufacturing needed?}
  \item{Why was it successful?}
  \item{Specific reasons = principles}
  \item{How can these be used to battle uncertainty?}
  \item{Lean Startup}
  \item{Unknown environment}
  \item{Digitalization}
  \item{If it does, it could be used for competitive advantage and better predicitons}
\end{itemize}

\fixme{Sources that could be used}

\cite{2014PhDT82H} \cite{Janes2015Guide} \cite{boes2014agile}

\section{Scope}
\label{section:scope}

The scope of the literature review will be existing literature on lean software projects, comparing these to find similarities if there are any.

Scope of the empirical study will be one lean software development project. The project will be studied from the point of view of developers, customers and end-users. The thesis is limited to one case study.

\section{Structure of the Thesis}
\label{section:structure}

This section presents the structure of the thesis.\\

\fixme{Write these a bit smoother once the structure is done}

Chapter \ref{chapter:background} presents the origin of the lean philosophy. It goes through the development of lean principles from manufacturing to software development.

Chapter \ref{chapter:litterature} covers the existing literature on lean software development. In this chapter experiences of previous lean software projects are analyzed and compared. This is done in order to find some common trends or best practices to use in projects. These common trends and best practices are compared in order to later compare them with the findings of the empirical study, which are presented in chapter \ref{chapter:discussion}.

Chapter \ref{chapter:methods} goes through the methods used in the empirical study. The participants and their roles are presented as well as the practical arrangements regarding interviews.

Chapter \ref{chapter:empirical} presents the results of the empirical study. The data is analyzed and the results of that analysis are presented.

Chapter \ref{chapter:discussion} discusses the results gathered from the empirical study and their implications. This is done by analyzing the relationship between the data gathered from the empirical study and the literature review presented int chapter \ref{chapter:litterature}.

Chapter \ref{chapter:conclusions} presents the conclusions of this thesis and suggestions for further research.


\input{2literature.tex}

 % -*- root: thesis.tex -*-

\chapter{Methods}
\label{chapter:mehods}

In this section I will describe how I conducted the empirical study.

\begin{itemize}
  \item{Interviews}
  \begin{itemize}
    \item{2-3 project members}
    \item{2-3 customers}
    \item{2-3 end-users}
    \item{1-1.5 hours each (5-10min for end-users)}
  \end{itemize}
  \item{Semi-structured interviews}
\end{itemize}



\input{4results.tex}

 % -*- root: thesis.tex -*-

\chapter{Discussion}
\label{chapter:discussion}

Here I will discuss how the findigs from my empirical work relate to the litterary review. What are the similarities and differences when comparing what the litterature says and what the interviews showed.



 % -*- root: thesis.tex -*-

\chapter{Conclusions}
\label{chapter:conclusions}

Here I mention the most important findings of the discussion section and the litterary review section.

I also point out how this research can be used in the future and what it's limitations are. (e.g. only one case study)

2 pages 



% Load the bibliographic references
% ------------------------------------------------------------------
% You can use several .bib files:
% \bibliography{thesis_sources,ietf_sources}
\bibliography{sources,Remote}


% Appendices go here
% ------------------------------------------------------------------
% If you do not have appendices, comment out the following lines
\appendix
\input{appendices.tex}

% End of document!
% ------------------------------------------------------------------
% The LastPage package automatically places a label on the last page.
% That works better than placing a label here manually, because the
% label might not go to the actual last page, if LaTeX needs to place
% floats (that is, figures, tables, and such) to the end of the
% document.
\end{document}
