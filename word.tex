\documentclass[12pt,a4paper,oneside,pdftex]{report}
\usepackage[utf8]{inputenc}
\usepackage{lmodern}
\usepackage[OT1]{fontenc}
\usepackage[finnish,swedish,english]{babel}


\usepackage[square,sort&compress,numbers]{natbib}
\usepackage{eurosym}

\usepackage{verbatim}

\usepackage{longtable}

\usepackage{subfigure}

\usepackage[medium]{titlesec}

\usepackage{tikz}
% You also need to specify which TikZ libraries you use
\usetikzlibrary{positioning}
\usetikzlibrary{calc}
\usetikzlibrary{arrows}
\usetikzlibrary{decorations.pathmorphing,decorations.markings}
\usetikzlibrary{shapes}
\usetikzlibrary{patterns}

% From Samwise, edited by JLarja. Fixme can be used to wrote something that should be fixed later. 
% It is better than just writing something to comments, since it is shown in the output file. If 
% mydraft is declared (in documentclass), text will be shown bold between two FIXMEs. If not, text 
% will be shown normally.
\newcommand{\fixme}[1]{#1}
% Draft command is used in the cover page to show the date of the
% draft (assumed to be today)
\newcommand{\DRAFT}{}
\DeclareOption{mydraft}{
\renewcommand{\fixme}[1]{\textsc{\small !Fixme} \textbf{#1}\textsc{\small Fixme!}}
}
\ProcessOptions

\newcommand{\TITLE}{Using lean to manage uncertainty in software development projects}
\newcommand{\FTITLE}{}
\newcommand{\STITLE}{}
\newcommand{\SUBTITLE}{A consulting perspective}
\newcommand{\FSUBTITLE}{}
\newcommand{\SSUBTITLE}{}
\newcommand{\DATE}{December 1, 2015}
\newcommand{\FDATE}{1. joulukuuta 2015}
\newcommand{\SDATE}{Den 1 December 2015}

% Supervisors and instructors
% ------------------------------------------------------------------
% If you have two supervisors, write both names here, separate them with a
% double-backslash (see below for an example)
% Also remember to add the package option ``twosupervisors'' or
% ``twoinstructors'' to the aalto-thesis package so that the titles are in
% plural.
% Example of one supervisor:
%\newcommand{\SUPERVISOR}{Professor Antti Ylä-Jääski}
%\newcommand{\FSUPERVISOR}{Professori Antti Ylä-Jääski}
%\newcommand{\SSUPERVISOR}{Professor Antti Ylä-Jääski}
% Example of twosupervisors:
\newcommand{\SUPERVISOR}{Professor Marjo Kauppinen}
\newcommand{\FSUPERVISOR}{Professori Marjo Kauppinen}
\newcommand{\SSUPERVISOR}{Professor Marjo Kauppinen}

% If you have only one instructor, just write one name here
\newcommand{\INSTRUCTOR}{Suvi Uski (?)}
\newcommand{\FINSTRUCTOR}{Suvi Uski (?)}
\newcommand{\SINSTRUCTOR}{Suvi Uski (?)}
% If you have two instructors, separate them with \\ to create linefeeds
% \newcommand{\INSTRUCTOR}{Olli Ohjaaja M.Sc. (Tech.)\\
%  Elli Opas M.Sc. (Tech)}
%\newcommand{\FINSTRUCTOR}{Diplomi-insinööri Olli Ohjaaja\\
%  Diplomi-insinööri Elli Opas}
%\newcommand{\SINSTRUCTOR}{Diplomingenjör Olli Ohjaaja\\
%  Diplomingenjör Elli Opas}

% If you have two supervisors, it is common to write the schools
% of the supervisors in the cover page. If the following command is defined,
% then the supervisor names shown here are printed in the cover page. Otherwise,
% the supervisor names defined above are used.
\newcommand{\COVERSUPERVISOR}{Professor Marjo Kauppinen, Aalto University}

% The same option is for the instructors, if you have multiple instructors.
% \newcommand{\COVERINSTRUCTOR}{Olli Ohjaaja M.Sc. (Tech.), Aalto University\\
%  Elli Opas M.Sc. (Tech), Aalto SCI}


% Other stuff
% ------------------------------------------------------------------
\newcommand{\PROFESSORSHIP}{Software Engineering and Business}
\newcommand{\FPROFESSORSHIP}{Ohjelmistotuotanto ja -liiketoiminta}
\newcommand{\SPROFESSORSHIP}{Programvaruproduktion och -affärsverksamhet}
% Professorship code is the same in all languages
\newcommand{\PROFCODE}{T-76}
\newcommand{\KEYWORDS}{lean, lean software, lean software project, service creation, agile, LSC}
\newcommand{\FKEYWORDS}{}
\newcommand{\SKEYWORDS}{}
\newcommand{\LANGUAGE}{English}
\newcommand{\FLANGUAGE}{Englanti}
\newcommand{\SLANGUAGE}{Engelska}

% Author is the same for all languages
\newcommand{\AUTHOR}{Oskar Ehnström}

% Abstracts

\newcommand{\ABSTRACTEN}{

The current literature on lean software development is derived from the literature on lean manufacturing. These principles can be applied as various practices in software projects depending on the context. The principles hold well for the general domain of software development.

Lean enables dealing with uncertainty, which has been found to be crucial when dealing with changing requirements. Software development often deals with uncertain requirements as well as new and unfamiliar technology. Whether lean could be used to deal with uncertainty in both requirements and technology has not been studied.

This thesis studies whether lean principles work as such for projects where the technology domain is unfamiliar to the project team. The study finds that the principles hold for projects facing uncertainty in both requirements and technology.

The study uses one software project as a case study. As such, general conclusions about the subject can no be drawn. However, the results suggest that lean could be a useful tool in handling uncertainty caused by unfamiliar technology in software projects.

}

\newcommand{\ABSTRACTFI}{}
\newcommand{\ABSTRACTSV}{}




\begin{document}

\title{\TITLE}
\AUTHOR

\chapter{Abstract}
\ABSTRACTEN

 % -*- root: thesis.tex -*-
\chapter{Introduction}
\label{chapter:intro}

\section{Background}
\label{section:background}

Futurice is a software consult agency developing digital solutions for their clients. Their methodology has alwasy been agile and has long focused on the overall experience of the end-users. They aim to take their customers through the whole lifecycle of a product, from ideation to lifecycle management. The process Futurice uses for this holistic approach is called Lean Service Creation (LSC).


\section{Research problem and question}
\label{section:problem}

The research problem is defined as follows:\\

\textit{How does a new service creation project that uses new and untested technology differ from a traditional service creation project that uses familiar technologies and concepts?}\\

To investigare this problem three research questions have been set up in table \ref{tbl:questions}.


\begin{table}
  \begin{tabular}{p{200pt} | p{70pt} | p{70pt}}
    Question & Litterature review & Empirical study \\
    \hline
    What are the currently available frameworks for new service creation? & x & \\
    What are the success factors of a new service creation project involving novel technology? &  & x \\
    What are the challenges when executing a new service creation project using novel technology? &  & x \\
  \end{tabular}
  \caption{Research questions and their respective sections}
  \label{tbl:questions}
\end{table}

\section{Scope}
\label{section:scope}




\section{Structure of the Thesis}
\label{section:structure}

The thesis is split into two distinct sections. First, chapter \ref{chapter:litterature} covers service design and innovation in general. Second, chapter \ref{chapter:empirical} covers the findings of the innovation culture at Futurice and how exposure to new technology has affected their new service creation work.




 % -*- root: thesis.tex -*-

\chapter{Background}
\label{chapter:background}

\fixme{This might be scaled down and moved to introduction.}

\section{The origin of lean thinking}
\label{section:leanorigins}


\fixme{This section will be about the history of lean and how it came to be applied to software engineering}

Even though the traditional wisdom is that the Japanese car manufacturers had a significant advantage over western competitors due to the lean methodologies they implemented there is some controversy over whether this was in fact the case. Dybá \& Sharp argue that by examining the facts and taking automation into account the Japanese did not have a superior organizational advantage. \cite{Dyba2012WhatS}

Shingo \cite{Shingo1989Study}

\section{Lean versus Agile}
\label{section:leanvsagile}

\fixme{This section will discuss the differences between lean and agile}

\fixme{
Lean should be thought of a set of principles rather than practices. This article has some excellent points and trends to talk about.\cite{Poppendieck2012Lean}
}


 % -*- root: thesis.tex -*-

\chapter{Literature review}
\label{chapter:litterature}

In this section I will present the existing literature and what general guidelines or conclusions it presents. I will refer to the main concepts presented in books and compare those findings to case studies that have implemented the relevant concepts in order to find some commonalities if there are any.

I will also present the process for finding and analyzing the material I have chosen.

\section{Literature overview}
\label{section:litoverview}

\begin{itemize}
  \item{How was the literature review conducted?}
  \begin{itemize}
    \item{Scholar}
    \item{Scholar through related articles of known good articles}
    \item{Key articles then their sources}
    \item{Same author, look for more}
    \item{Same conference/journal, look for more}
  \end{itemize}
  \
\end{itemize}

Keywords: \textit{lean software development, lean software management, lean project management, digital service creation, service-dominant, design thinking, new service development}

\section{Lean software principles}
\label{section:leansoftwareprinciples}

In this section I will present the principles for lean software development as presented in \cite{poppendieck2003lean} and subsequent articles.

\section{Comparison of past lean projects}
\label{section:pastleanprojects}

In this section I will compare articles regarding case studies focusing on lean software development/management.

Two case studies. This is an early attempt to test lean software development. The article covers an experiment set up to test the validity of using lean principles in software development. \cite{Middleton2001Lean}

Timberline Inc. case study. Probably the first case of adopting lean principles to software development.\cite{Middleton2005Lean}

BBC Worldwide case study. The gist of it is that the performance of the team improved when adopting lean practices, but there were some challenges in fitting the the lean principles with the rest of the company.\cite{Middleton2012Lean}


\section{Research problem and question}
\label{section:problem}

This chapter will end with the research problem and questions.

The research problem is defined as follows:\\

\textit{What are the commonly accepted and used lean software development practices and how do they change when working with new and unfamiliar technology?}\\

To investigate this problem three research questions have been set up in table \ref{tbl:questions}.


\begin{table}
  \begin{tabular}{p{200pt} | p{70pt} | p{70pt}}
    Question & Literature review & Empirical study \\
    \hline
    What are the currently available best practices for lean software projects? & x & \\
    Which of these best practices need to be adapted when working with new technology? &  & x \\
    How do these best practices need to adapted? &  & x \\
    Which best practices remain valid? &  & x \\
  \end{tabular}
  \caption{Research questions and their respective sections}
  \label{tbl:questions}
\end{table}

 % -*- root: thesis.tex -*-

\chapter{Methods}
\label{chapter:methods}

% Introduction

This chapter presents the methods used to gather and analyze the data of this study. First, the research problem is reiterated and the motivation for the used methods are presented. Second, grounded theory, the chosen method of analysis, is presented. Third, the way data was collected through interviews is presented. Fourth, the participants and their backgrounds are disclosed. Fifth, the process of analysis is explained using grounded theory. Finally, the methods are critically analyzed and their validity and credibility are evaluated.

\section{Methodological solution}
\label{section:methodsolution}

% Research problem and why this method suits it

The research problem defined in section \ref{section:problem} aims to understand the lean software development process. Interviews are the most straightforward way to try to understand the underlying culture of lean in this context. A proven method is needed to analyze the data from the interviews. Grouping by themes found in literature, such as the principles presented in chapter \ref{chapter:literature}, is one option. A better option is to analyze the data as such, without preconceived notions of the results. The grounded theory is well suited for this purpose.

\section{Method of analysis}
\label{section:methodanalysis}

% Present grounded theory

Grounded theory was first presented by Glaser and Strauss in 1967. Their field was sociology and the aim of grounded theory was to enable qualitative researchers to formulate theories which would be relevant in that field. The book was aimed at sociologists, but the authors recognized that the theory could be useful for any domain looking to understand social phenomena using qualitative data. \cite{glaser1967discovery}

As the name suggests, grounded theory is grounded in the data. This means that when using grounded theory one starts by gathering data and analyzing it. This is different from the traditional approach in science where you first form a hypothesis and then verify or disprove the hypothesis by empirical study. This roundabout way of forming a theory has led some to calling the hypothesis formed by grounded theory as ``reverse engineered''. \cite{lazar2010research}

Various forms of empirical data can serve as the starting point for grounded theory research. In this research the source material is interviews, but GT can also be used with ethnography, observation, and case studies among others. \cite{lazar2010research}

The GT method is performed in four steps. The first step is coding. This is where the researchers identify phenomenon in the source material and gives them names or codes. In GT coding is done as open coding. This refers to the fact that the coding is done with an open mind, letting the concepts emerge from the source material and not from outside influences. The second step is developing concepts. This is where codes that describe similar phenomenon are grouped together to form concepts. In the third step, categories are grouped together to form categories. Finally, a theory is formed based on the causal connections between concepts and categories. \cite{lazar2010research}


\section{Data collection}
\label{section:methoddata}

% haastikset, semi structured

The empirical study was conducted as a series of interviews. Five interviews were conducted, each lasting between 50 and 65 minutes.

The interviews were conducted as semi-structured interviews. Two sets of questions were prepared, one set for Futurice employees and one for customer employees. The Futurice employee questions were modified slightly to better suit the expertise of the participant. The questions may be found in Appendix \ref{appendix:questions}. During the interviews participants were encouraged to answer with their own interpretations. Follow up questions were also asked based on the direction of the answers the participants gave.

The questions were prepared based on the lean principles presented in chapter \ref{chapter:literature}. Subjects were asked about previous experiences with software projects and with lean methods. The questions explored project based work as well as organizational culture. Questions were also left intentionally up for interpretation in order to map the subject's own interpretations and avoid unnecessary bias by the interviewer.

\fixme{Add questions appendix}

\section{Participants}
\label{section:methodparticipants}

The participants were from Futurice (3 participants) and Futurice's customer organizations (2 participants).

Participants were asked to sign a letter of informed consent before the interview. The interviews were recorded in audio format and detailed written notes were done based on the recordings. The interviewer agreed to archive the recordings responsibly and not share the recordings with any third party. The letter of informed consent can be studied in appendix \ref{appendix:letter}.

\fixme{Add letter of informed consent as appendix}

\section{Process of analysis}
\label{section:methodprocess}

% miten teemat syntyivät

Coding and categorization of the interviews was done using the Atlas.ti software. The written notes from the interviews were imported and codified. Then networks of concepts were modeled using the network functionality of the software.

Categories were chosen based on the data form the interviews. The data presented some clear themes which were identified. Some in-vivo categories were also present in the data, but most were general themes. It is safe to assume that the themes were influenced by the existing literature on lean, both based on the set of questions and the bias of the researcher.

\fixme{Write this again when the actual analysis has been done}

\section{Method evaluation}
\label{section:methodevaluation}

% credibility etc.
True objectivity is very hard to attain. The preconceptions and background of the author were recognized and taken into account during the analysis process and special care was taken to analyze the data sensitively. However, it can still be assumed that the analysis is not entirely objective, as the researcher always affects the process.

Subjects were asked for clarifying questions and viewpoints when they seemed to veer too far off course. Clarifications about statements were asked to confirm the assumed meaning of answers which could have been interpreted by the subjects as ``the right answer'' and thus confirmed. The subjects were, however, informed repeatedly that there were no wrong answers and as such they could feel safe answering any way they wished. The credibility of this study is at a satisfactory level, although it is safe to assume that the researcher influenced the interview situation.

\fixme{TODO: Ask subjects to read and ``confirm'' analysis}

The low number of subjects and organizations studied makes the findings of the study specific and not generalizable. However, the results of the study serve as a good baseline for further research. The analysis of this study is valid for a specific set and can be used as a base for further study.

The interviews were conducted in one round. Ideally GT is done iteratively so that once one round of interviews has been analyzed the theory can be confirmed or refined by a second round of interviews. This study lacked the resources to conduct several iterations of interviews, which has to be taken into account when assessing the validity of the data.

Themes from the existing literature were clearly present in the answers of the subject. Some of the questions were specifically formulated to seek out answers related to the existing literature, but even those that were not supported the principals present in the existing literature. Existing literature supports the findings of the study.



 % -*- root: thesis.tex -*-

\chapter{Results}
\label{chapter:empirical}

In this section I will present the results of the empirical study. This section contains the raw data of the conducted interviews. This section does not analyze or compare the results with the literature.


 % -*- root: thesis.tex -*-

\chapter{Discussion}
\label{chapter:discussion}

Here I will discuss how the findings from my empirical work relate to the literary review. What are the similarities and differences when comparing what the literature says and what the interviews showed.

\section{Lean Service Creation}
\label{section:lsc}

\fixme{Lean Service Creation is about build measure learn and create something new. Tie this into the problem and focus on the new part. Using new devices maybe?}

Lean Service Creation is based on the ideas of ``The Lean Startup'' as described by Eric Reis in his 2011 book.\cite{ries2011lean}

\fixme{How do I get a source for this? Interview?}


 % -*- root: thesis.tex -*-

\chapter{Conclusions}
\label{chapter:conclusions}

Here I mention the most important findings of the discussion section and the literary review section.

I also point out how this research can be used in the future and what it's limitations are. (e.g. only one case study)

2 pages 


\chapter{References}

\end{document}
