\newcommand{\TITLE}{Using lean to manage uncertainty in software development projects}
\newcommand{\FTITLE}{}
\newcommand{\STITLE}{}
\newcommand{\SUBTITLE}{A consulting perspective}
\newcommand{\FSUBTITLE}{}
\newcommand{\SSUBTITLE}{}
\newcommand{\DATE}{December 1, 2015}
\newcommand{\FDATE}{1. joulukuuta 2015}
\newcommand{\SDATE}{Den 1 December 2015}

% Supervisors and instructors
% ------------------------------------------------------------------
% If you have two supervisors, write both names here, separate them with a
% double-backslash (see below for an example)
% Also remember to add the package option ``twosupervisors'' or
% ``twoinstructors'' to the aalto-thesis package so that the titles are in
% plural.
% Example of one supervisor:
%\newcommand{\SUPERVISOR}{Professor Antti Ylä-Jääski}
%\newcommand{\FSUPERVISOR}{Professori Antti Ylä-Jääski}
%\newcommand{\SSUPERVISOR}{Professor Antti Ylä-Jääski}
% Example of twosupervisors:
\newcommand{\SUPERVISOR}{Professor Marjo Kauppinen}
\newcommand{\FSUPERVISOR}{Professori Marjo Kauppinen}
\newcommand{\SSUPERVISOR}{Professor Marjo Kauppinen}

% If you have only one instructor, just write one name here
\newcommand{\INSTRUCTOR}{Suvi Uski (?)}
\newcommand{\FINSTRUCTOR}{Suvi Uski (?)}
\newcommand{\SINSTRUCTOR}{Suvi Uski (?)}
% If you have two instructors, separate them with \\ to create linefeeds
% \newcommand{\INSTRUCTOR}{Olli Ohjaaja M.Sc. (Tech.)\\
%  Elli Opas M.Sc. (Tech)}
%\newcommand{\FINSTRUCTOR}{Diplomi-insinööri Olli Ohjaaja\\
%  Diplomi-insinööri Elli Opas}
%\newcommand{\SINSTRUCTOR}{Diplomingenjör Olli Ohjaaja\\
%  Diplomingenjör Elli Opas}

% If you have two supervisors, it is common to write the schools
% of the supervisors in the cover page. If the following command is defined,
% then the supervisor names shown here are printed in the cover page. Otherwise,
% the supervisor names defined above are used.
\newcommand{\COVERSUPERVISOR}{Professor Marjo Kauppinen, Aalto University}

% The same option is for the instructors, if you have multiple instructors.
% \newcommand{\COVERINSTRUCTOR}{Olli Ohjaaja M.Sc. (Tech.), Aalto University\\
%  Elli Opas M.Sc. (Tech), Aalto SCI}


% Other stuff
% ------------------------------------------------------------------
\newcommand{\PROFESSORSHIP}{Software Engineering and Business}
\newcommand{\FPROFESSORSHIP}{Ohjelmistotuotanto ja -liiketoiminta}
\newcommand{\SPROFESSORSHIP}{Programvaruproduktion och -affärsverksamhet}
% Professorship code is the same in all languages
\newcommand{\PROFCODE}{T-76}
\newcommand{\KEYWORDS}{lean, lean software, lean software project, service creation, agile, LSC}
\newcommand{\FKEYWORDS}{}
\newcommand{\SKEYWORDS}{}
\newcommand{\LANGUAGE}{English}
\newcommand{\FLANGUAGE}{Englanti}
\newcommand{\SLANGUAGE}{Engelska}

% Author is the same for all languages
\newcommand{\AUTHOR}{Oskar Ehnström}

% Abstracts

\newcommand{\ABSTRACTEN}{

The current literature on lean software development is derived from the literature on lean manufacturing. These principles can be applied as various practices in software projects depending on the context. The principles hold well for the general domain of software development.

Lean enables dealing with uncertainty, which has been found to be crucial when dealing with changing requirements. Software development often deals with uncertain requirements as well as new and unfamiliar technology. Whether lean could be used to deal with uncertainty in both requirements and technology has not been studied.

This thesis studies whether lean principles work as such for projects where the technology domain is unfamiliar to the project team. The study finds that the principles hold for projects facing uncertainty in both requirements and technology.

The study uses one software project as a case study. As such, general conclusions about the subject can no be drawn. However, the results suggest that lean could be a useful tool in handling uncertainty caused by unfamiliar technology in software projects.

}

\newcommand{\ABSTRACTFI}{}
\newcommand{\ABSTRACTSV}{}

