 % -*- root: thesis.tex -*-

\newcommand{\TITLE}{Lean Software Development as a Service}
\newcommand{\FTITLE}{}
\newcommand{\STITLE}{}
\newcommand{\SUBTITLE}{Understanding lean software projects}
\newcommand{\FSUBTITLE}{}
\newcommand{\SSUBTITLE}{}
\newcommand{\DATE}{December 1, 2015}
\newcommand{\FDATE}{1. joulukuuta 2015}
\newcommand{\SDATE}{Den 1 December 2015}

% Supervisors and instructors
% ------------------------------------------------------------------
% If you have two supervisors, write both names here, separate them with a
% double-backslash (see below for an example)
% Also remember to add the package option ``twosupervisors'' or
% ``twoinstructors'' to the aalto-thesis package so that the titles are in
% plural.
% Example of one supervisor:
%\newcommand{\SUPERVISOR}{Professor Antti Ylä-Jääski}
%\newcommand{\FSUPERVISOR}{Professori Antti Ylä-Jääski}
%\newcommand{\SSUPERVISOR}{Professor Antti Ylä-Jääski}
% Example of twosupervisors:
\newcommand{\SUPERVISOR}{Professor Marjo Kauppinen}
\newcommand{\FSUPERVISOR}{Professori Marjo Kauppinen}
\newcommand{\SSUPERVISOR}{Professor Marjo Kauppinen}

% If you have only one instructor, just write one name here
\newcommand{\INSTRUCTOR}{Suvi Uski (?)}
\newcommand{\FINSTRUCTOR}{Suvi Uski (?)}
\newcommand{\SINSTRUCTOR}{Suvi Uski (?)}
% If you have two instructors, separate them with \\ to create linefeeds
% \newcommand{\INSTRUCTOR}{Olli Ohjaaja M.Sc. (Tech.)\\
%  Elli Opas M.Sc. (Tech)}
%\newcommand{\FINSTRUCTOR}{Diplomi-insinööri Olli Ohjaaja\\
%  Diplomi-insinööri Elli Opas}
%\newcommand{\SINSTRUCTOR}{Diplomingenjör Olli Ohjaaja\\
%  Diplomingenjör Elli Opas}

% If you have two supervisors, it is common to write the schools
% of the supervisors in the cover page. If the following command is defined,
% then the supervisor names shown here are printed in the cover page. Otherwise,
% the supervisor names defined above are used.
\newcommand{\COVERSUPERVISOR}{Professor Marjo Kauppinen, Aalto University}

% The same option is for the instructors, if you have multiple instructors.
% \newcommand{\COVERINSTRUCTOR}{Olli Ohjaaja M.Sc. (Tech.), Aalto University\\
%  Elli Opas M.Sc. (Tech), Aalto SCI}


% Other stuff
% ------------------------------------------------------------------
\newcommand{\PROFESSORSHIP}{Software Engineering and Business}
\newcommand{\FPROFESSORSHIP}{Ohjelmistotuotanto ja -liiketoiminta}
\newcommand{\SPROFESSORSHIP}{Programvaruproduktion och -affärsverksamhet}
% Professorship code is the same in all languages
\newcommand{\PROFCODE}{T-76}
\newcommand{\KEYWORDS}{lean, lean software, lean software project, service creation, agile, LSC}
\newcommand{\FKEYWORDS}{}
\newcommand{\SKEYWORDS}{}
\newcommand{\LANGUAGE}{English}
\newcommand{\FLANGUAGE}{Englanti}
\newcommand{\SLANGUAGE}{Engelska}

% Author is the same for all languages
\newcommand{\AUTHOR}{Oskar Ehnström}

% Abstracts

\newcommand{\ABSTRACTEN}{

Lean thinking has proven to be a successful principle for modern software development projects, just as lean manufacturing was for the auto industry.

Existing research shows that the principles of lean translate well into organizations working on traditional software products.

The impact of lean by vendors working as software consultants in other industries has been largely ignored by current research.

This research looks at the principles used by a modern software consulting company that delivers software for customer organizations using lean software development.

Representatives of both the vendor and the customers are interviewed and the data gathered from these are analyzed using grounded theory.

The results from this study can form a basis for further study in the best practices of lean software development in consulting software development companies.

}

\newcommand{\ABSTRACTFI}{
1. intro - context, missä ollaan, johdanto...
2. tutkimusongelma - whats the problem?
3. mitä tiede on tehnyt tämän ongelman ratkaisemisekti? mikä on gappi?
4. kerro miten ratkaiset ongelman... minä ratkaisen näin...
5. metodi - millä minä tutkin?
6. tutkimuksen impakti - esittele tulokset tai miten nämä vaikuttaa olemassa olevaan tieteseen/käytäntöön.
}
\newcommand{\ABSTRACTSV}{

}

