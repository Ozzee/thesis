 % -*- root: thesis.tex -*-

\chapter{Background}
\label{chapter:background}

\section{The origin of lean thinking}
\label{section:leanorigins}


\fixme{This section will be about the history of lean and how it came to be applied to software engineering}

Even though the traditional wisdom is that the Japanese car manufacturers had a significant advantage over western competitors due to the lean methodologies they implemented there is some controcersy over wether this was in fact the case. Dyb� \& Sharp argue that by examining the facts and taking automation into account the Japanese did not have a superior organizational advantage. \cite{Dyba2012WhatS}

\section{Lean versus Agile}
\label{section:leanvsagile}

\fixme{This section will discuss the differences between lean and agile}

\fixme{
Lean should be thought of a set of principles rather than practices. This article has some excellent points and trends to talk about.\cite{Poppendieck2012Lean}
}

\fixme{End with something about everything being services nowdays and lean/agile being developer focused. The continue to...}

\section{Lean Service Creation}
\label{section:lsc}

\fixme{Lean Service Creation is about build measure learn and create somehing new. Tie this into the problem and focus on the new part. Using new devices maybe?}

Lean Service Creation is based on the ideas of ``The Lean Startup'' as described by Eric Reis in his 2011 book.\cite{ries2011lean}

\fixme{How do I get a source for this? Interview?}