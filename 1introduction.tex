 % -*- root: thesis.tex -*-

 % Tiimalasti rakenne

\chapter{Introduction}
\label{chapter:intro}

\section{Background}
\label{section:background}

\begin{itemize}
  \item{Lean dictionary definition}
  \item{Where does lean come from?}
  \item{Why was lean manufacturing needed?}
  \item{Why was it successful?}
  \item{Specific reasons = principles}
  \item{How can these be used to battle uncertainty?}
  \item{Lean Startup}
  \item{Unknown environment}
  \item{Digitalization}
  \item{If it does, it could be used for competitive advantage and better predicitons}
\end{itemize}

Even though the traditional wisdom is that the Japanese car manufacturers had a significant advantage over western competitors due to the lean methodologies they implemented there is some controversy over whether this was in fact the case. Dybá \& Sharp argue that by examining the facts and taking automation into account the Japanese did not have a superior organizational advantage. \cite{Dyba2012WhatS}

\fixme{
Lean should be thought of a set of principles rather than practices. This article has some excellent points and trends to talk about.\cite{Poppendieck2012Lean}
}

\fixme{Sources that could be used}

\cite{2014PhDT82H} \cite{Janes2015Guide} \cite{boes2014agile}

\section{Research problem and question}
\label{section:problem}

The companies presented in section \ref{section:comparison} are all companies writing their own software for a familiar domain. However, these companies still mention organizational challenges in implementing lean principles. Especially big companies may see moving to lean principles as too much of a disturbance in their business and something that would be hard to convince all stakeholders to get on board with.

Big companies often can, and will, use subcontractors to outsource parts of their work. These subcontractors can be smaller companies, and as such they have the ability to more easily use a lean approach to their work. Could there be a way to utilize a strategic partnership with skilled subcontractors who utilize lean principles in order to gain the benefits of lean thinking without having to fundamentally change the whole company?

This question is an interesting one. If small and lean partners could be utilized to achieve a lean approach to projects it could clear the way for an overall more lean organization. At the very least, these lean partners could be utilized in projects where competition favors a lean approach.

The research problem is thus defined as follows:\\

\textit{How does Futurice utilize lean principles to deliver value for their customers?}\\

To investigate this problem three research questions have been set up in table \ref{tbl:questions}.


\begin{table}
  \begin{tabular}{p{200pt} | p{70pt} | p{70pt}}
    Question & Literature review & Empirical study \\
    \hline
    What are the established lean software development principles? & x & \\
    How do software vendors utilize these principles in their work? & x & x \\
    How do lean projects affect the project environment of the customer? &  & x \\
  \end{tabular}
  \caption{Research questions and their respective sections}
  \label{tbl:questions}
\end{table}

\section{Scope}
\label{section:scope}

The scope of the literature review will be existing literature on lean software projects, comparing these to find similarities if there are any.

Scope of the empirical study will be one lean software development project. The project will be studied from the point of view of developers, customers and end-users. The thesis is limited to one case study.

\section{Structure of the Thesis}
\label{section:structure}

This section presents the structure of the thesis.\\

\fixme{Write these a bit smoother once the structure is done}

Chapter \ref{chapter:background} presents the origin of the lean philosophy. It goes through the development of lean principles from manufacturing to software development.

Chapter \ref{chapter:litterature} covers the existing literature on lean software development. In this chapter experiences of previous lean software projects are analyzed and compared. This is done in order to find some common trends or best practices to use in projects. These common trends and best practices are compared in order to later compare them with the findings of the empirical study, which are presented in chapter \ref{chapter:discussion}.

Chapter \ref{chapter:methods} goes through the methods used in the empirical study. The participants and their roles are presented as well as the practical arrangements regarding interviews.

Chapter \ref{chapter:empirical} presents the results of the empirical study. The data is analyzed and the results of that analysis are presented.

Chapter \ref{chapter:discussion} discusses the results gathered from the empirical study and their implications. This is done by analyzing the relationship between the data gathered from the empirical study and the literature review presented int chapter \ref{chapter:litterature}.

Chapter \ref{chapter:conclusions} presents the conclusions of this thesis and suggestions for further research.
