 % -*- root: thesis.tex -*-

 % Tiimalasti rakenne

\chapter{Introduction}
\label{chapter:intro}

Lean thinking has been gaining traction in the software development industry in recent years. Books like ``The Lean Startup'' and ``Lean Software Development: An Agile Toolkit'' have taught us how to take lean practices that were initially used in manufacturing and use them to create value in software development and service design. Organizations have, at this point, had the time to implement lean practices and try the suggested methods in practice. This makes it possible to compare findings and analyze what the common best practices and problems are.

Although uncertainty is always a factor in software projects, new and unfamiliar technology introduces even more of it and can potentially lead to budget overruns or even project failure. Understanding how to adjust accepted lean practices to these projects would enable projects to perform more reliable estimates and avoid introducing unnecessary risk when striving for high rewards using new technology.

\section{Scope}
\label{section:scope}

The scope of the literature review will be existing literature on lean software projects, comparing these to find similarities if there are any.

Scope of the empirical study will be one lean software development project. The project will be studied from the point of view of developers, customers and end-users. The thesis is limited to one case study.

\section{Structure of the Thesis}
\label{section:structure}

This section presents the structure of the thesis.\\

\fixme{Write these a bit smoother once the structure is done}

Chapter \ref{chapter:background} presents the origin of the lean philosophy. It goes through the development of lean principles from manufacturing to software development.

Chapter \ref{chapter:litterature} covers the existing literature on lean software development. In this chapter experiences of previous lean software projects are analyzed and compared. This is done in order to find some common trends or best practices to use in projects. These common trends and best practices are compared in order to later compare them with the findings of the empirical study, which are presented in chapter \ref{chapter:discussion}.

Chapter \ref{chapter:methods} goes through the methods used in the empirical study. The participants and their roles are presented as well as the practical arrangements regarding interviews.

Chapter \ref{chapter:empirical} presents the results of the empirical study. The data is analyzed and the results of that analysis are presented.

Chapter \ref{chapter:discussion} discusses the results gathered from the empirical study and their implications. This is done by analyzing the relationship between the data gathered from the empirical study and the literature review presented int chapter \ref{chapter:litterature}.

Chapter \ref{chapter:conclusions} presents the conclusions of this thesis and suggestions for further research.
