 % -*- root: thesis.tex -*-

\chapter{Literature review}
\label{chapter:litterature}

In this section I will present the existing literature and what general guidelines or conclusions it presents. I will refer to the main concepts presented in books and compare those findings to case studies that have implemented the relevant concepts in order to find some commonalities if there are any.

I will also present the process for finding and analyzing the material I have chosen.

\section{Literature overview}
\label{section:litoverview}

\begin{itemize}
  \item{How was the literature review conducted?}
  \begin{itemize}
    \item{Scholar}
    \item{Scholar through related articles of known good articles}
    \item{Key articles then their sources}
    \item{Same author, look for more}
    \item{Same conference/journal, look for more}
  \end{itemize}
  \
\end{itemize}

Keywords: \textit{lean software development, lean software management, lean project management, digital service creation, service-dominant, design thinking, new service development}

\section{Lean software principles}
\label{section:leansoftwareprinciples}

In this section I will present the principles for lean software development as presented in \cite{poppendieck2003lean} and subsequent articles.

\section{Comparison of past lean projects}
\label{section:pastleanprojects}

In this section I will compare articles regarding case studies focusing on lean software development/management.

Two case studies. This is an early attempt to test lean software development. The article covers an experiment set up to test the validity of using lean principles in software development. \cite{Middleton2001Lean}

Timberline Inc. case study. Probably the first case of adopting lean principles to software development.\cite{Middleton2005Lean}

BBC Worldwide case study. The gist of it is that the performance of the team improved when adopting lean practices, but there were some challenges in fitting the the lean principles with the rest of the company.\cite{Middleton2012Lean}


\section{Research problem and question}
\label{section:problem}

This chapter will end with the research problem and questions.

The research problem is defined as follows:\\

\textit{What are the commonly accepted and used lean software development practices and how do they change when working with new and unfamiliar technology?}\\

To investigate this problem three research questions have been set up in table \ref{tbl:questions}.


\begin{table}
  \begin{tabular}{p{200pt} | p{70pt} | p{70pt}}
    Question & Literature review & Empirical study \\
    \hline
    What are the currently available best practices for lean software projects? & x & \\
    Which of these best practices need to be adapted when working with new technology? &  & x \\
    How do these best practices need to adapted? &  & x \\
    Which best practices remain valid? &  & x \\
  \end{tabular}
  \caption{Research questions and their respective sections}
  \label{tbl:questions}
\end{table}