 % -*- root: thesis.tex -*-

\chapter{Literature review}
\label{chapter:litterature}

This chapter presents the existing literature on lean software development. The methods used for gathering academic sources is presented, as well as a general overview on the subject. Finally existing literature on past lean software development projects are compared to the general principles of lean and to each other.

\section{Literature overview}
\label{section:litoverview}

This section presents the methods used to gather material for the literature review.\\

Google Scholar was the primary source for material on lean and lean software development. Searching the numerous databases in Google Scholar with the keywords mentioned in figure \ref{figure:keywords}. Once some suitable primary articles had been found these could be used to find related articles. By utilizing the related articles feature on Google Scholar more related articles could be found. The sources of known articles also led to other articles on the subject and especially to heavily cited articles that provide the foundation for many other articles. Searching for articles written by the authors of known articles also led to other articles that dealt with the subject. Conferences and journals that included some known articles also proved to include other similar articles and were a good source of material.

\begin{figure}[h]
  \label{figure:keywords}
  \begin{center}
    \begin{tabular}{| l |}
      \hline
      Search term \\
      \hline
      lean software development \\
      lean software management \\
      lean project management \\
      digital service creation \\
      lean \\
      \hline
     \end{tabular}
    \caption{Keywords used to find sources}
  \end{center}
\end{figure}


\section{Lean software principles}
\label{section:leansoftwareprinciples}

\fixme{In this section I will present the principles for lean software development as presented in \cite{poppendieck2003lean} and subsequent articles.}

\section{Comparison of past lean projects}
\label{section:pastleanprojects}

In this section I will compare articles regarding case studies focusing on lean software development/management.

Two case studies. This is an early attempt to test lean software development. The article covers an experiment set up to test the validity of using lean principles in software development. \cite{Middleton2001Lean}

Timberline Inc. case study. Probably the first case of adopting lean principles to software development.\cite{Middleton2005Lean}

BBC Worldwide case study. The gist of it is that the performance of the team improved when adopting lean practices, but there were some challenges in fitting the the lean principles with the rest of the company.\cite{Middleton2012Lean}


\section{Research problem and question}
\label{section:problem}

This chapter will end with the research problem and questions.

The research problem is defined as follows:\\

\textit{What are the commonly accepted and used lean software development practices and how do they change when working with new and unfamiliar technology?}\\

To investigate this problem three research questions have been set up in table \ref{tbl:questions}.


\begin{table}
  \begin{tabular}{p{200pt} | p{70pt} | p{70pt}}
    Question & Literature review & Empirical study \\
    \hline
    What are the currently available best practices for lean software projects? & x & \\
    Which of these best practices need to be adapted when working with new technology? &  & x \\
    How do these best practices need to adapted? &  & x \\
    Which best practices remain valid? &  & x \\
  \end{tabular}
  \caption{Research questions and their respective sections}
  \label{tbl:questions}
\end{table}